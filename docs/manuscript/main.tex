\documentclass[12pt]{article}
%
\usepackage{abstract,amsmath,amssymb,latexsym}
\usepackage{tikz-cd}
\usepackage{enumitem,epsf}
\usepackage{fullpage,tikz,float}
\usepackage[numbers]{natbib}
\usepackage[pdftex,colorlinks]{hyperref}

% locally defined macros
\usepackage{macros}

%%%%%%%%%%%%%%%%%%%%%%%%%%%%%%%%%%%%%%%%%%%%%%%%%%%%%%%%%%%%%

% Use the following for revealing TODOs and appendices
% Options are: \draftfalse or \drafttrue
\newif\ifdraft
\draftfalse

%%%%%%%%%%%%%%%%%%%%%%%%%%%%%%%%%%%%%%%%%%%%%%%%%%%%%%%%%%%%%

\title{
 \begin{minipage}[c]{1.05\textwidth}
 	\centerline{Towards a Continuous Hyperparameter}
 	\centerline{Representation for Neural Networks}
 \end{minipage}
}

\author{
	\vspace{1cm}
	William Guss\thanks{Email: wguss@berkeley.edu} \and
	Other Author\thanks{Email: other@berkeley.edu} \and 
	Other Author\thanks{Email: other@berkeley.edu}
}

\begin{document}

\maketitle
\thispagestyle{empty}

%% \vspace{-4cm}
%% \vspace*{\fill} {
%% \renewcommand{\abstractnamefont}{\normalfont\large\bfseries}
%% \renewcommand{\abstracttextfont}{\normalfont\normalfont}
%% \renewcommand{\baselinestretch}{1.0}
\begin{abstract}
%% \vspace{-0.25cm}
% \input{abstract}
\end{abstract}
%% }
%% \vspace*{\fill}

\ifdraft
\newpage
\input{todo}
\fi

\newpage
\thispagestyle{empty}

{\large
\topskip0pt
\vspace*{\fill}
\tableofcontents
\vspace*{\fill}
}

\newpage

\setcounter{page}{1}

% TODO: After planning phase move to sections.

\section{Planning \& Unorganized Results (*)}
\subsection{Motivation \& Goal}

\subsection{Questions/Hypothesis}

\subsection{Theory}

\subsubsection{Desired Results}

\subsubsection{Some Exposition}

\subsection{Experiments}

The following are a set of desired experiments to verify the newly proposed hyperparameter representation.

\subsection{Reading List}

\subsection{Related Notes}

\begin{itemize}
	\item \href{https://onedrive.live.com/edit.aspx/Documents/Machine%20Learning?cid=c05fef92a4e2f5cd&id=documents&wd=target%28Continuous%20Parameterization.one%7C577F6089-A9C8-40F1-A37C-474241ED5D20%2F%29
onenote:https://d.docs.live.net/c05fef92a4e2f5cd/Documents/Machine%20Learning/Continuous%20Parameterization.one#section-id={577F6089-A9C8-40F1-A37C-474241ED5D20}&end}{Continuous Hidden Dimension}
	\item \textbf{Some Thoughts on Local Search on Hidden Units.} Let $\scriptn$ be the $\mathfrak{n}$-discrete instatntiation of the following DFM
	\begin{equation*}
	\begin{tikzcd}
		\scripto: \boxed{\mathbb{R}^n} \arrow{r}{\mathfrak{d}} & \boxed{L^1(E(\gamma))} \arrow{r}{\mathfrak{f}} & \boxed{\mathbb{R}}
	\end{tikzcd}
	\end{equation*}
	where $E: \mathbb{R} \to \scriptl(\mathbb{R})$ is a function which parameterizes the domain over which the $\mathfrak{f}$-functional integrates.
	

	It was concluded in the last note that if $E(\gamma) = [0, \gamma] \in \scriptl(\mathbb{R})$ then we have the following problem for the piecewise constant
	parameterization of weights on $\mathfrak{f}, \mathfrak{d}.$ Let $F: \mathbb{R} \to \mathbb{R}$ be some loss function, and then computation of the local gradient ascent path
	gives
	\begin{equation*}
	\begin{aligned}
		\frac{\partial F}{\partial \gamma} &= \frac{d F}{d y^2} \frac{\partial y^2}{\partial \gamma} \\
		 &= \frac{d F}{d y^2} \cdot \left[\frac{\partial }{\partial \gamma}  \int_{[0, \gamma]} \sum_{k=1}^\infty  [\sigma \circ \mathfrak{d}(x)](u)  \chi_{k\cdot [0,1]}(u) W^1_{k}\ d\mu(u)   \right]_\mathfrak{n} \\
		 &= \frac{d F}{d y^2} \cdot \left[ \sum_{k=1}^\infty  [\sigma \circ \mathfrak{d}(x)](\gamma)  \chi_{k\cdot [0,1]}(\gamma) W^1_{k}    \right]_\mathfrak{n} \\
		 &= \frac{d F}{d y^2} \cdot  y^1_{\floor*{\gamma}} W^1_{\floor*{\gamma}}.
	\end{aligned}
	\end{equation*}

	In otherwords, gradient ascent on $F$ with respect to $\gamma$ will increase $\gamma$ if the error will decrease when the contribution of the last output neuron is increased (in magnitude); that is, if $\gamma' > \gamma$ then $(\gamma - \floor*{\gamma})$ increases, and thus $E$ decreases by virtue of the term 
	\begin{equation*}
		\int_{\floor*{\gamma}\cdot [0,1]} y^1(u) W^1_{\floor*{\gamma}}\ d\mu(u) = (\gamma - \floor*{\gamma}) y^1_{\floor*{\gamma}} W^1_{\floor*{\gamma}}
	\end{equation*} increasing. Searching over $\gamma$ is effecitvely the same as spending extra time changing the weight $W_{\floor*{\gamma}}^1$ using two linearly dependent parameters, $(\gamma - \floor*{\gamma})$ and $W_{\floor*{\gamma}}^1$, itself\footnote{An additonal conclusion is, at least by analogy, that local search on $E(\gamma)$ at any one place assumes that adjacent neurons have similar values }. 

	Thus we are led to the question: \emph{Is hyperparameter search a matter of immediate model accuracy or expected capacity for accuracy, and in that distinction, does optimizing hyperparameters with respect to model accuracy coorespond to optimization on model capactiy and visa versa?}  Let us examine this question in the following context. \

	Above, we noted that a local search on $\gamma$ decreased error in exactly the same fashion as standard gradient descent, but a step in $\gamma$ of more than integral amount can increase error. To see this let $k = \floor*{\gamma}$. When $\Delta \gamma > 1$ then the $(k+1)$th neuron is then "enabled" so-to-speak. However, this $(k+1)$th neuron may perform a computation that increases error and so in the next step of gradient descent $\Delta \gamma$ would be negative, retreating away from the added model capacity of a randomly intiialized $(k+1)$th neuron. That is not to say that $\gamma$ might not increase again, repeating the process, or in the limit of such oscilations the update $ W^1_{k+1} -\alpha\partial E/\partial W^1_{k+1} \to W^1_{k+1}$, will eventually contribute to model accuracy, but relying on these dynamics as a result with no guarentees of convergence is questionable. Despite the fact that $\scriptn$ may need additional model capacity\footnote{There are functions which are unlearnable without a sufficient number of neurons for example.}, local search on capacity with respect to accuracy may not yield the required capacity to increase accuracy in the limit.


	Baring that local search doesn't necisarrily yield the desired properties, we might now consider a global search.
	The more general setting is of course considering $E$ as a function of variable support geometry. In particular let $E: C^{\infty}_*(\mathbb{R}) \to \scriptl(\mathbb{R})$ so that $f \mapsto \text{supp}(|f|/2 + f/2).$ Not that $C^{\infty}_*(\mathbb{R})$ is the set of infinitely differentiable functions which vanish on at most a $\mu$-null set. Then computation of a gradient descent step becomes a variational problem
	\begin{equation*}
		\begin{aligned}
			\frac{\delta F}{\delta f} &= \frac{\partial F}{\partial y^2} \frac{\delta y^2}{\delta f}  \\
			&= \frac{\partial F}{\partial y^2} \left[\frac{\delta}{\delta f} \int_{E(f)} y^1 \omega_1\ d\mu \right]_{\mathfrak{n}} \\
			&= \frac{\partial F}{\partial y^2} \left[\frac{\delta}{\delta f} \lim_{\kappa \to \infty}\int_{\reals} {\sigma}(\kappa f) y^1 \omega_1\ d\mu \right]_{\mathfrak{n}}
		\end{aligned}
	\end{equation*}

	Now we attempt to compute the functional derivative of integration; that is, let $J_\kappa[f] = \int_{\reals} \sigma(\kappa f) y^1 \omega_1\ d\mu$.
	Then
	\begin{equation*}
		\begin{aligned}
			\frac{\delta J}{\delta f} &= \lim_{\epsilon \to 0} \lim_{\kappa \to \infty} \frac{J_\kappa[f + \epsilon \phi] - J_\kappa[f] }{\epsilon} \\
			&= \lim_{\kappa \to \infty} \left[\frac{d}{d\epsilon} J_\kappa[f + \epsilon \phi]\right]_{\epsilon = 0} \\
			&= \lim_{\kappa \to \infty} \left[\int_{\mathbb{R}} \frac{d}{d\epsilon} \sigma(\kappa(f + \epsilon \phi))y^1\omega_1\ d\mu \right]_{\epsilon = 0} \\
			&= \lim_{\kappa \to \infty} \left[\int_{\mathbb{R}}  \sigma'(\kappa(f + \epsilon \phi)) \kappa \phi  y^1\omega_1\ d\mu \right]_{\epsilon = 0} \\
			&=  \int_{\mathbb{R}}  \lim_{\kappa \to \infty} \sigma'(\kappa f) \kappa \phi  y^1\omega_1\ d\mu. \\
			&=  \int_{\mathbb{R}} \delta(f(u)) \phi(u)  y^1(u)\omega_1(u)\ d\mu(u).
			= \sum_{z \in Z(f)} \phi(z) y^1(z) \omega_1(z)
		\end{aligned}
	\end{equation*}
	where $Z(f)$ is the set of zeroes of $f$. Thus we yield a functional gradient ascent step via the linear approximation
	\begin{equation*}
	\begin{aligned}
		J[\phi] &= J[f] + \frac{\delta J}{\delta f}[\phi -f] + \frac{1}{2t} \left\|\phi - f\right\|^2 \\
		0&= 0 +  \frac{\delta J}{\delta f} + \frac{\phi - f}{t} \\
		\phi &= f - t \left(\psi \mapsto \sum_{z \in Z(f)} \psi(z) y^1(z) \omega_1(z)\right).
	\end{aligned}
	\end{equation*}
	In addition to the previous update rule, we come to the redundant conclusion that $\delta J/\delta f|_{f=\Gamma} = 0$ for $\Gamma$ with no zeros, and therefore when $\Gamma < 0$ we have only found a minimum for $J$. In any case, the gradient in the hard limit of $k$ is zero at every point at which $f(u)$ is non-zero. Using a soft limit $\kappa \not \to \infty$ we get an ascent direction in the continuum of our initial search on $\gamma$l that is
	\begin{equation*}
		\phi = f - \sigma'(\kappa f) \kappa \phi \sum_{k=1}^\infty \chi_{k\cdot [0,1]} \left[\sigma \circ \delta(x)\right] W_k^1.
	\end{equation*}
	However, we face the same question as to whether optimziation on the loss function would yield ascent in the direction of random vectors for the purpose of capacity and not by their similarity actions to values which decrease error.

	   \item \textbf{Fuzzy Vector Spaces as a Solution to Continuous Parameterization of Hidden Dimension.} As we saw in the previous  o


\end{itemize}

\subsection{Timeline}

T






% \input{related}
% \section{Setting}
% \input{skeletons}
% \input{results}
% \input{compkerspa}
% \input{proofs}
% \input{discussion}

\ifdraft
\appendix
% \input{appendix}
\fi

% \subsection*{Acknowledgments}

\bibliographystyle{plainnat}
\bibliography{bib}

\end{document}
